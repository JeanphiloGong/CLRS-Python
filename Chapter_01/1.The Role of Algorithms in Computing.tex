\documentclass{book}
\usepackage{amsmath}  % for mathematical symbols and environments
\usepackage{listings}

\title{Introduction to Algorithms Notes}
\author{Jean philo Gong}
\date{\today}  % or specify a date

\begin{document}

\maketitle

\tableofcontents  % Generates a table of contents

\chapter{The Role of Algorithms in Computing Notes}

\section{Algorithms}

\paragraph{An example of an algorithm is as follows:}
\begin{align*}
    &\text{Input: A sequence of } n \text{ numbers } (a_1, a_2, \ldots, a_n). \\
    &\text{Output: A permutation (reordering) } (a_1', a_2', \ldots, a_n') \\
    &\quad \text{such that } a_1' \leq a_2' \leq \ldots \leq a_n'.
\end{align*}

\subsection{What kinds of problems are solved by algorithms}

\subsection{Data structures}

\paragraph{Definition:}
\begin{align*}
    &\text{A data structure is a way to store and organize data in order to} \\
    &\text{facilitate access and modifications.}
\end{align*}

\subsection{Technique}

\section{Algorithms as a technology}
% other content





\chapter{Getting Started}
% content for this chapter
\section{Insertion sort}
    \begin{align*}
        &\text{Input: A sequence of } n \text{ numbers } (a_1, a_2, \ldots, a_n). \\
        &\text{Output: A permutation (reordering) } (a_1', a_2', \ldots, a_n') \\
        &\quad \text{such that } a_1' \leq a_2' \leq \ldots \leq a_n'.
    \end{align*}
\subsection*{Methods}
\subsection{Insertion Sort Algorithm}

The insertion sort algorithm can be broken down into the following steps:

\begin{enumerate}
    \item Define a function to perform the insertion sort operation.
    \item Loop starts from the second element.
    \item Store the current element as the key.
    \item Initialize \( j \) as the element just before \( i \).
    \item Move elements that are greater than the key to one position ahead of their current position.
    \item Place the key in its correct position.
\end{enumerate}

\subsection{Code Implementation}

The Python code for the insertion sort algorithm is given below:

\begin{lstlisting}[language=Python]
def insertion_sort(arr):
    for i in range(1, len(arr)):
        key = arr[i]
        j = i - 1
        while j >= 0 and key < arr[j]:
            arr[j + 1] = arr[j]
            j -= 1
        arr[j + 1] = key
\end{lstlisting}

\section{Analyzing algotirhms}

\section{Designing algorithms}


\chapter{Growth of Functions}
% content for this chapter
\section{Asymptotic notation}

\section{Standard notations and common functions}




\chapter{Divide-and-Conquer}
% content for this chapter
\section{The maximum-subarray problem}

\section{Strassen's algorithm for matrix multiplication}

\section{The substitution method for solving recurrences}

\section{The recursion-tree method for solving recurrences}

\section{The master method for solving recurrences}

\section{Proof of the master theorem}



\chapter{Probabilistic Analysis and Randomized Algorithms}
% content for this chapter





\chapter{Heapsort}
% content for this chapter




\chapter{Quicksort}
% content for this chapter




\chapter{Sorting in Linear Time}
% content for this chapter




\chapter{Medians and Order Statistics}
% content for this chapter




\chapter{Elementary Data Structures}
% content for this chapter




\chapter{Hash Tables}
% content for this chapter




\chapter{Binary Search Trees}
% content for this chapter




\chapter{Red-Black Trees}
% content for this chapter




\chapter{Augmenting Data Structures}
% content for this chapter




\chapter{Dynamic Programming}
% content for this chapter




\chapter{Greedy Algorithms}
% content for this chapter




\chapter{Amortized Analysis}
% content for this chapter




\chapter{B-Trees}
% content for this chapter




\chapter{Fibonacci Heaps}
% content for this chapter




\chapter{Van Emde Boas Trees}
% content for this chapter




\chapter{Data Structures for Disjoint Sets}
% content for this chapter




\chapter{Graph Algorithms}
% content for this chapter




\chapter{Minimum Spanning Trees}
% content for this chapter





\chapter{Single-Source Shortest Paths}
% content for this chapter





\chapter{All-Pairs Shortest Paths}
% content for this chapter




\chapter{Maximum Flow}
% content for this chapter




\chapter{Multithreaded Algorithms}
% content for this chapter




\chapter{Matrix Operations}
% content for this chapter




\chapter{Linear Programming}
% content for this chapter




\chapter{Polynomials and the FFT}
% content for this chapter




\chapter{Number-Theoretic Algorithms}
% content for this chapter




\chapter{String Matching}
% content for this chapter




\chapter{Computational Geometry}
% content for this chapter




\chapter{NP-Completeness}
% content for this chapter




\chapter{Approximation Algorithms}
% content for this chapter




\chapter{Mathematical Background}
% content for this chapter




\chapter{Problems, Hints, and Solutions}
% content for this chapter




\end{document}